% Преамбула.
\documentclass[a4paper, 12pt, oneside, openany, book]{ncc}

\ChapterPrefixStyle{header,toc}
\PnumPrototype{999}
\usepackage[headings]{ncchdr} % Линейка в колонтитуле для стиля headings

\usepackage{rus_ruby_book}

% Тело документа.
\begin{document}

\author{Алекcандр Круглов}
\title{Ruby}
\date{\today}
\bookeditor{Ruby 2.0.0p247}
\maketitle

\frontmatter
  \include{author}
  \tableofcontents

\mainmatter

\part{Основы}
  \include{intro}
  \include{standart}
  \include{identificate}
  \include{expression}
  \include{oop}

\part{Описание классов}
  \chapter{Числа}
$$
\xymatrix{
Numeric \: (+ \: Comparable) \ar[d]\ar[dr]\ar[drr]\ar[drrr] &&& \\
Complex & Rational & Float & Integer \ar[d]\ar[dr] & \\
&&& Fixnum & Bignum }
$$

Для математических расчетов в Ruby определен модуль Math.

\input{numeric}
\input{integer}
\input{float}
\input{rational}
\input{complex}
\input{math}

  \chapter{Текст}

\input{string}
\input{regexp}
\input{encode}

  \chapter{Составные объекты}

\textbf{Добавленные модули: Enumerable}

\input{array.tex}
\input{hash.tex}
\input{range.tex}
\input{enumerator.tex}
\input{enumerable.tex}

  \include{object}
  \include{subroutine}
  \include{random}
  \include{datetime}
  \include{datatype}

\part{Работа программы}
  \include{execution}
  \chapter{Чтение и запись данных}

\input{stream.tex}
\input{file.tex}
\input{dir.tex}
\input{filestat.tex}

  \include{args}
  \include{library}
  \include{exception}
  \include{test}
  \include{thread}
  \include{security}

\appendix

\addcontentsline{toc}{part}{Приложения}
\part*{Приложения}
  \include{appbin}
  \include{appregexp}
  \include{appformat}
  \include{appassign}
  \include{appencode}
  \include{apppack}
  \include{appdatetime}
  \include{appio}
  \include{appfile}

\backmatter
  \addcontentsline{toc}{part}{Заключение}
  \chapter*{Заключение}
  \input{conclusion}

\end{document}
